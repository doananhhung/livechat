\subsection{UserModule}
\label{subsec:user-module}

\subsubsection{Mục đích}

\texttt{UserModule} là module nền tảng, chịu trách nhiệm quản lý toàn bộ vòng đời và dữ liệu của một thực thể \texttt{User} trong hệ thống. \texttt{User} ở đây đại diện cho một nhân viên hỗ trợ (agent) có tài khoản để đăng nhập và sử dụng dashboard. Module này cung cấp các dịch vụ cốt lõi để các module khác (đặc biệt là \texttt{AuthModule}) có thể tương tác với dữ liệu người dùng một cách an toàn và nhất quán.

\subsubsection{Các Thành phần Chính (Files)}

\begin{description}
    \item[\texttt{user.entity.ts}] Định nghĩa cấu trúc dữ liệu cho thực thể \texttt{User} bằng TypeORM. Nó mô tả các cột trong bảng \texttt{users} của cơ sở dữ liệu, bao gồm thông tin định danh (email, password), thông tin cá nhân (fullName), và các cờ trạng thái liên quan đến bảo mật (isTwoFactorAuthenticationEnabled).
    
    \item[\texttt{user.service.ts}] Chứa toàn bộ logic nghiệp vụ để thao tác với dữ liệu người dùng. Service này tương tác trực tiếp với database repository, che giấu sự phức tạp của việc truy vấn CSDL khỏi các controller. Nó đảm bảo các quy tắc nghiệp vụ (ví dụ: không tạo người dùng trùng email) được tuân thủ.
    
    \item[\texttt{user.controller.ts}] Cung cấp các API endpoint (qua giao thức HTTP) để người dùng đã xác thực có thể quản lý thông tin tài khoản của chính họ. Tất cả các endpoint trong controller này đều được bảo vệ bởi \texttt{JwtAuthGuard}, yêu cầu người dùng phải gửi một JWT hợp lệ.
    
    \item[\texttt{user.module.ts}] Đóng gói tất cả các thành phần trên. Nó đăng ký \texttt{User} entity với TypeORM và quan trọng nhất là export \texttt{UserService}, cho phép các module khác trong ứng dụng (như \texttt{AuthModule}) có thể inject và sử dụng service này.
\end{description}

\subsubsection{Phân tích các Phương thức Quan trọng}

\paragraph{\texttt{user.service.ts}}
Service này chứa logic nghiệp vụ đa dạng, từ việc quản lý CRUD cơ bản đến các quy trình bảo mật phức tạp liên quan đến refresh token và xác thực hai yếu tố (2FA).

\begin{itemize}
    \item \texttt{create(dto)}: Tạo một người dùng mới trong một transaction để đảm bảo tính toàn vẹn dữ liệu.
    
    \item \texttt{findOneById(id)} và \texttt{findOneByEmail(email)}: Là các phương thức truy vấn cơ bản, đóng vai trò nền tảng cho các nghiệp vụ khác như xác thực và kiểm tra quyền.
    
    \item \texttt{requestEmailChange(id, newEmail, password)}: Xử lý yêu cầu thay đổi email của người dùng. Phương thức này thực hiện một chuỗi các bước bảo mật quan trọng: xác thực mật khẩu hiện tại của người dùng, kiểm tra xem email mới đã được sử dụng hay chưa, và cuối cùng là cập nhật email mới trong một transaction.
    
    \item \texttt{updateProfile(id, dto)}: Cập nhật các thông tin cá nhân không nhạy cảm của người dùng như họ tên, avatar.
    
    \item \texttt{activate(id)} và \texttt{deactivate(id)}: Quản lý trạng thái hoạt động của tài khoản người dùng.
    
    \item \textbf{Quản lý Refresh Token:}
    \begin{itemize}
        \item \texttt{setCurrentRefreshToken(options)}: Là trung tâm của cơ chế Refresh Token Rotation. Nó nhận một refresh token mới, băm và lưu vào CSDL, đồng thời tìm và xóa refresh token cũ đã được sử dụng (nếu có) trong cùng một transaction.
        \item \texttt{verifyRefreshToken(token, id)}: Xác thực một refresh token được cung cấp. Nó so sánh token này với các token đã được băm trong CSDL và kiểm tra xem token có còn hạn sử dụng hay không. Nó sẽ tự động xóa token hết hạn.
        \item \texttt{removeAllRefreshTokensForUser(id)}: Xóa tất cả các refresh token của một người dùng, thường được sử dụng khi người dùng đăng xuất khỏi tất cả các thiết bị.
    \end{itemize}
    
    \item \textbf{Quản lý Xác thực hai yếu tố (2FA):}
    \begin{itemize}
        \item \texttt{turnOnTwoFactorAuthentication(id, secret)}: Kích hoạt 2FA. Phương thức này mã hóa và lưu trữ 2FA secret, tạo ra 10 mã khôi phục (recovery codes), sau đó băm và lưu các mã này vào CSDL. Nó trả về các mã khôi phục dưới dạng plaintext cho người dùng để họ lưu trữ.
        \item \texttt{turnOffTwoFactorAuthentication(id)}: Tắt 2FA, đồng thời xóa 2FA secret và tất cả các mã khôi phục liên quan khỏi CSDL để đảm bảo an toàn.
    \end{itemize}
\end{itemize}

\paragraph{\texttt{user.controller.ts}}
Controller này được thiết kế để chỉ xử lý các yêu cầu từ người dùng đã được xác thực (\texttt{@UseGuards(JwtAuthGuard)}). Nó đóng vai trò là cổng giao tiếp cho phép người dùng tự quản lý tài khoản của mình.

\begin{itemize}
    \item \texttt{GET /me}: Endpoint để người dùng lấy thông tin hồ sơ của chính họ. Một điểm bảo mật quan trọng là phương thức này đã được lập trình để **loại bỏ trường \texttt{passwordHash}** ra khỏi đối tượng trả về cho client, tránh làm lộ thông tin nhạy cảm.
    
    \item \texttt{PATCH /me}: Cho phép người dùng cập nhật thông tin cá nhân (họ tên, avatar). Tương tự như \texttt{GET /me}, nó cũng loại bỏ \texttt{passwordHash} khỏi response.
    
    \item \texttt{POST /request-email-change}: Endpoint để người dùng khởi tạo yêu cầu thay đổi địa chỉ email của họ, yêu cầu cung cấp email mới và mật khẩu hiện tại để xác thực.
    
    \item \texttt{DELETE /me}: Endpoint cho phép người dùng tự vô hiệu hóa tài khoản của mình. Thay vì xóa vĩnh viễn, nó gọi đến service \texttt{deactivate} để chuyển trạng thái của người dùng.
\end{itemize}